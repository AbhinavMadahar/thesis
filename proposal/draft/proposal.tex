\documentclass[12pt]{article}

\usepackage{amsmath}
\usepackage[OT1]{fontenc}
\usepackage{fontspec}
\usepackage{epsfig}
\usepackage{graphics}
\usepackage{times}

\renewcommand\rmdefault{cmr}
\renewcommand\sfdefault{cmss}
\renewcommand\ttdefault{cmtt}

\newfontfamily\devanagari{Noto Sans Devanagari}[Script=Devanagari]

% <http://psl.cs.columbia.edu/phdczar/proposal.html>:
%
% The standard departmental thesis proposal format is the following:
%        30 pages
%        12 point type
%        1 inch margins all around = 6.5   inch column
%        (Total:  30 * 6.5   = 195 page-inches)
%
% For letter-size paper: 8.5 in x 11 in
% Latex Origin is 1''/1'', so measurements are relative to this.

\topmargin      0.0in
\headheight     0.0in
\headsep        0.0in
\oddsidemargin  0.0in
\evensidemargin 0.0in
\textheight     9.0in
\textwidth      6.5in

\title{{\bf Doctoral Thesis Proposal}}
\author{ {\bf Abhinav Madahar $\circ$ {\normalfont {\devanagari अभिनव मदहर}}}  \\
Department of Computer Science \\
(Your University Here!) \\
{\small abhinavmadahar@tbd.edu}}
\date{December 1, 2023}

\begin{document}
\pagestyle{plain}
\pagenumbering{roman}
\maketitle

\pagebreak
\vspace*{\fill}
\noindent Speech is my hammer, bang the world into shape.

\noindent Yasiin Bey, 1999
\vspace*{\fill}

\pagebreak
\vspace*{\fill}
\noindent Speech is my hammer, bang the world into shape.

\noindent Yasiin Bey, 1999
\vspace*{\fill}

\pagebreak
\vspace*{\fill}
\noindent
When submitting an application to graduate school, you can include auxiliary documents.
There are no restrictions on what you can upload.
Your application to graduate school is reviewed by professors in your field.
This means that, in effect, I have been given a megaphone through which I can send any message I want to American academia.

\vspace{1cm}

\noindent
This is my message.

\vspace*{\fill}

\pagebreak
\begin{abstract}

\end{abstract}

\pagebreak
\tableofcontents
\pagebreak

\cleardoublepage
\pagenumbering{arabic}


\begin{footnotesize}
\bibliographystyle{plain}
\bibliography{string,itu,rfc,i-d}
\end{footnotesize}

\end{document}


