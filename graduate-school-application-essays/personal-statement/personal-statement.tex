\documentclass{amsart}

\usepackage{enumitem}
\usepackage[OT1]{fontenc}
\usepackage{fontspec}
\usepackage{geometry}
\usepackage{parskip}

\geometry{margin=2cm}

\newfontfamily\devanagari{Noto Sans Devanagari}[Script=Devanagari]

\title{Personal Statement}
\author{Abhinav Madahar $\circ$ {\normalfont {\devanagari अभिनव मदहर}}}
\date{\today}

\begin{document}

\maketitle

The discourse on university admissions in the United States focuses on characteristics such as race, gender, sexual orientation, and so forth.
I would like to add a new element to the discussion: caste.
In Hinduism, people are divided into four castes: brahmins, kshatriyas, vaishyas, and finally shudras.
There is a fifth group, dalits, people so detested that they are excluded from the caste system itself; this is where I was born.

Though not obvious, caste exists as a reality in the United States through two mechanisms.
To understand the first mechanism, consider that caste is a socio-religious construction rather than a political or legal one, so it exists anywhere where there is a Hindu community.
As a result of large-scale migration from India to the US, there are now Hindu communities in the US, and, though most Hindu Americans do not explicitly demonstrate caste discrimination, an implicit bias is inevitable, just as implicit biases exist for race, gender, and other factors.
The second avenue through which caste exerts itself is through intergenerational heredity; people in lower castes tend to be underprivileged because their parents were.
Growing up, my parents struggled to afford even the bare necessities; I had to sleep on the floor as a child because they could not afford to buy me a mattress.

Though it has not gained full recognition in American society, caste is slowly being recognized; Brown and Barnard recently added it to their protected classes, and I expect more universities to do so as time goes on.
In the Opinion of the Court for Students for Fair Admissions, Chief Justice John Roberts noted the tendency of population grouping to ignore intra-group diversity; using his example, treating all Latinos as a single group would prefer an incoming class which is 15\% Mexican over one which is 10\% Latino from a variety of Latin American countries even though the later is clearly more diverse.
I argue that universities should explicitly ask for caste to those applicants to whom it is applicable during the application process alongside other demographic factors like race and gender.
In the same way that grouping all Latinos together ignores diversity of national origin, grouping all Indians together ignores diversity of caste; there is an ocean of a difference between an incoming class of brahmins and an incoming class of dalits.
Unless universities consider caste in the same way as they do race and gender, there will likely be a tendency to favour upper-caste applicants, which would go against the mission of increasing diversity and centralizing members of marginalized communities.

Some simple back-of-the-envelope calculations show that, if I were to be admitted, then I would likely be the only dalit student in the incoming class.
Were I to be accepted, I would try to use my presence to increase awareness of caste to the other students, particularly because caste is poorly understood in the United States.

Thank you for reading.

\end{document}
